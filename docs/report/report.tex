\documentclass[english]{uzhpub}
\usepackage[T1]{fontenc}
\usepackage[latin9]{inputenc}

\begin{document}

%% Titelei
\title{Master Project: Clustermeister}

\subtitle{Report}

\author{Daniel Spicar, Thomas Ritter}

\date{\today}

\maketitle

\section{Motivation}

\subsection{What is Clustermeister?}
Clustermeister provides a framework for easy code execution and testing on remote and distributed Java Virtual Machines (JVM). Specifically it provides utilities to facilitate remote code deployment scenarios and an API to execute code on remote JVMs.

\subsection{Problems Addressed}
Testing code on a dynamically provisioned cluster or in the cloud is in most cases a hassle for Java/Scala developers. The code needs to be packaged, cluster nodes have to be allocated, the allocated nodes have to be found, the packaged code needs to be deployed to all the nodes and usually the JVM on the node has to be restarted. This process is often managed using a variety of tools, ranging from SSH scripts to custom cloud APIs and clustering frameworks. The whole process is time-consuming to set up, manage and run. Another issue is that usually the connection from the developer machine to the cluster is slow, so transferring files from the developer machine to all the nodes is potentially slow. 

Clustermeister tries to solve this issue by providing tools to set up nodes easily and fast.


\subsection{Provided Services}

\begin{itemize}
\item Deployment of JPPF (TODO: what is JPPF?) nodes on (virtual) machines requiring only minimal configuration.
\item Provisioning of Amazon EC2 instances provided by jClouds (TODO: jcoulds intro).
\item Parallel and distributed code execution via a Java ExecutorService interface or a native JPPF interface.
\item Dynamic classloading allowing for rapid re-execution of client code without manual re-deployment.
\item Addressable nodes for code execution on specific nodes.
\item Easy deployment of dependencies using maven repository dependency resolution provided by Sonatype Aether (TODO: aether intro).
\end{itemize}

\subsection{Supported Environments}
\begin{itemize}
\item Execution in JVMs on the local machine.
\item TORQUE (PBS) Clusters.
\item Amazon Web Services Elastic Compute Cloud (EC2).
\end{itemize}

\section{Organisation}



\section{Architecture}

\subsection{Terminology}

\begin{description}
\item[Clustermeister Node] A node can be interpreted as a JVM that executes code. Nodes are running on local or remote Clustermeister instances. A single instance can host several nodes.
\item[Clustermeister Instance] An instance is a physical or virtual machine generally running in a cluster or a cloud computation service such as Amazon EC2.
\end{description}

\subsection{Clustermeister Modules}

Clustermeister consists of two main modules: Provisioning and API.

The major advantage of this separation of concerns is, that it enables the user to set up and deploy nodes and instances at the beginning of a development or computation session. Using this set-up the user can repeatedly execute code without the need for re-deployment of changed code. This speeds up developing and testing code in a distributed environment.

\subsection{Clustermeister Provisioning}
The provisioning module is accessible via a command line interface (CLI) and is responsible for deployment of the Clustermeister infrastructure. This enables provisioning of instances and nodes, deployment of dependencies and dynamic classloading.

Clustermeister provisioning is used to set up distributed nodes for a development or computation session.

\subsection{The Clustermeister API}
Allows the user to send jobs and tasks to Clustermeister nodes for execution and access to the computation results. The computations are executed asynchronously, in parallel and in a distributed manner. Clustermeister API supports execution of Serializable and JVM executable code. This supports not only Java but also bytecode compatible programming languages such as Scala.

The Clustermeister API is used to access the nodes and features provided by the Clustermeister provisioning module during a development or computation session.

\subsection{Toplology}

TODO

\section{Implementation}



\section{User Manual}

% lstset{language=Java, numbers=left, showspaces=false, frame=single, breaklines=true}

\end{document}
